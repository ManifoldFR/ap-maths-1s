\documentclass[10pt]{beamer}
\usepackage{geometry}
\usepackage{mathtools}
\usepackage{amssymb}
\usepackage{amsthm}
\usepackage{amsfonts}
\usepackage{graphicx}

\usetheme{Frankfurt}

\usepackage{fontspec}
\usepackage{polyglossia}
\setdefaultlanguage{french}
\frenchspacing

\renewcommand\epsilon\varepsilon
\renewcommand\phi\varphi
\newcommand{\N}{\mathbb N}
\newcommand{\Z}{\mathbb Z}
\newcommand{\R}{\mathbb R}
\newcommand{\Q}{\mathbb Q}
\newcommand{\CC}{\mathbb C}
\newcommand{\K}{\mathbb K}
\DeclarePairedDelimiter{\intcc}{[\![}{]\!]}

\title{Atelier d'informatique}
\subtitle{\textbf{Épisode I:} Introduction}

\begin{document}

\begin{frame}
	\titlepage
	
	<< \textit{Un programme informatique fait ce que vous lui avez dit de faire, pas ce que vous voulez qu'il fasse.} >> — Troisième loi de Greer
\end{frame}

\begin{frame}
	\frametitle{Informatique: késako ?}
	
	Le mot \textit{informatique} est la contraction des mots \textit{information} et \textit{automatique} : il s'agit donc de la science du traitement automatique de l'information.
	\pause
	
	Un \textit{ordinateur} est la concrétisation de cette notion, une machine qui traite automatiquement des informations données en entrée, selon un \textit{programme informatique} qui dicte comment procéder.
\end{frame}

\begin{frame}[fragile]
	\frametitle{À la découverte de Python}
	
	\begin{itemize}
	\item<1-> Démarrez sur votre bureau le programme \textbf{Pyzo}.
	
	\item<2-> \textbf{Pyzo} est un environnement de développement pour le langage de programmation Python. Il inclut une \textbf{console} (ou \textit{shell}, ou \textit{interpréteur}), où sont entrées les instructions à exécuter immédiatement, et une zone où écrire des scripts.
	
	\end{itemize}
	\pause[3]
	
	\begin{figure}
		\centering
		\includegraphics[height=0.5\textheight]{screen_pyzo.png}
		\caption{Pyzo.}
	\end{figure}
	
	
\end{frame}

\begin{frame}[fragile]
	\frametitle{Votre première manip'}
	\begin{block}{\textbf{Exercice 1} Calculs élémentaires}
		On va commencer par se familiariser avec les opérations arithmétiques de base que peut faire Python.
		
		\begin{description}[<+->]
		\item[Addition] Entrer \verb|2+2| dans la console, appuyer sur \verb|Entrée| pour exécuter votre saisie. Vérifier que ça fait \verb|4|.
		\item[Soustraction] Entrer \verb|4-3|. Vérifier que l'on trouve \verb|1|.
		\item[Produit] Taper \verb|2*3|. Vérifier que l'on trouve \verb|6|.
		\item[Division] Taper \verb|12/3|. Vérifier que l'on trouve \verb|4|. Taper \verb|4/5|. Vérifier que l'on trouve \verb|0.8|.
		\item[...de CM1] Taper \verb|13//4|, et \verb|13%4|. Compléter la division posée suivante:
			\[
			\begin{array}{c|c}
			13      & 4\\ \cline{2-2}
			\ldots  & \ldots
			\end{array}
			 \]
			Que remarquez-vous ?
		\item[Puissances] Taper \verb|2**3|. Quel est le résultat ? Et celui de \verb|3**2| ?
		\end{description}
	\end{block}
\end{frame}

\begin{frame}[fragile]

Mais Python est bien plus puissant qu'une simple \textit{Casio collège}...
\pause
	\begin{block}{\textbf{Exercice 2} Variables}
		Une \textit{variable} est la donnée d'un emplacement mémoire où une valeur est stockée, et d'un nom.
		\pause
		
		Le fait d'affecter une valeur à une variable s'appelle une << \textit{affectation} >>. En Python, on fait ça avec une expression de la forme \verb|x=...|	
		\pause
		
		On peut également réaffecter une variable en recyclant simplement son nom.
		\pause
		
		Pour afficher la valeur d'une variable, on peut demander à Python de l'évaluer en tapant son nom.
		\pause
		
		\begin{itemize}[<+->]
		\item Créer une variable $x$ qui a la valeur $3$. Est-ce qu'on peut écrire \verb|3=x| ?
		
		\item Créer une variable $y$ qui a la valeur $7x$. Modifier la valeur de $x$. Cela modifie-t-il la valeur de $y$ ?
		
		\item Rajouter $1$ à $x$ en la réaffectant, vérifier en évaluant. Même question avec $0.5$.
		
		\item Que font \verb|x+=1|, \verb|x-=1|, \verb|x*=2| ou encore \verb|x/=2| ?
		
		\item Entrer l'instruction \verb|x,y=y,x|. Quelles sont alors les valeurs de $x$ et de $y$ ?
		\end{itemize}
		\pause[8]
	\end{block}
\end{frame}

\begin{frame}[fragile]
	\textbf{Remarque} Dans un langage de programmation un peu plus \textit{kasher}, on déclare et on affecte séparément une variable. Mais Python permet de déclarer une variable directement par affectation.
	\pause
	
	\vspace{1em}
	\textbf{À propos des types}
	
	La fonction \verb|type|, appliquée à une variable $x$ via l'instruction \verb|type(x)|, renvoie le type de la variable $x$. Essayez sur plusieurs valeurs: \verb|type(0)|, \verb|type(0.5)|, \verb|type(type)|, et \verb|type(0.)|.
	\pause
	
	Les types \verb|int| et \verb|float| servent respectivement à représenter les nombres entiers et les nombres réels à virgule dans Python. On peut faire les opérations arithmétiques vues plus haut avec elles, même quand les variables en jeu sont d'un type et de l'autre.
	
\end{frame}

\begin{frame}[fragile]
	\begin{block}{\textbf{Exercice 3} Chaînes de caractères}
	Un nouveau type de variable important: le type \verb|str|, pour \textit{string} ou \textit{chaîne de caractères}, en français. Définis explicitement en tapant un message encadré par des guillemets simples (\verb|' '|) ou doubles (\verb|" "|).
	\pause
	
	\begin{itemize}[<+->]
	\item Définir une variable \verb|Bonjour| prenant la valeur $0$.
	
	\item Que fait \verb|print(Bonjour)| ? Et les instructions \verb|print("Bonjour")| et \verb|print('Bonjour')| ?
	
	\item Comparer \verb|type(Bonjour)|, \verb|type('Bonjour')| et \verb|type("Bonjour")|.
	
	\item Évaluer \verb|len("Bonjour")|. À quoi cela correspond-t-il ?
	
	\item Afficher \verb|J'aime la tartiflette| et  \verb|Il dit: "Bonjour !"|. Attention à utiliser des types de guillemets différents !
	
	\item Que se passe-t-il quand on affiche une chaîne contenant \verb|\n| ?
	
	\item Définir deux chaines de caractères \verb|x| et \verb|y|: que fait \verb|print(x, y)| ? Est-ce que ça marche aussi si \verb|x| est une variable de type \verb|int| ?
	
	\item Évaluer la valeur de \verb|x+y|, l'afficher via \verb|print|.
	
	\item Afficher << \verb|1/100 est petit| >> en remplaçant \verb|1/100| par sa valeur. On utilisera \verb|"{} est petit".format()| où \verb|x| est la valeur voulue.
	
	\end{itemize}
	\end{block}
\end{frame}

\begin{frame}[fragile]
	\frametitle{Premiers programmes}
	
	On va maintenant basculer sur la zone d'écriture de scripts. Un programme est une succession d'instructions qui sont effectuées lorsqu'il est exécuté.
	\pause
	
	\begin{block}{\textbf{Exercice 4}}
	\begin{itemize}[<+->]
	\item Écrire un programme qui stocke la chaîne << \verb|Bonjour| >> dans une variable \verb|x|, puis l'affiche, puis affiche << \verb|Bonjour Bonjour| >>.
	
	\item Écrire un programme qui demande à l'utilisateur d'entrer un texte, affiche << \verb|Vous avez entré:| >> suivi du texte entré. On utilisera la fonction \verb|input|:	\begin{semiverbatim}texte = input()\end{semiverbatim}
	demande, à son exécution, une chaîne de caractère à l'utilisateur, puis la stocke dans la variable \verb|texte|.
	
	\item Écrire un programme qui demande à l'utilisateur un nombre, le stocke dans une variable \verb|x|, puis affiche << \verb|f(x) = | >> suivi de la valeur de $\frac{1}{1+x^2}$. On pourra utiliser \verb|input|, convertir son entrée (initialement de type \verb|str|) en un entier via le constructeur \verb|int|.
	
	\end{itemize}
	\end{block}
	
\end{frame}

\begin{frame}[fragile]
	\frametitle{Structures conditionnelles}
	
	On peut demander à un programme de traiter différemment ses données selon les valeurs des variables introduites. Pour cela, on utilise une structure conditionnelle, qui prend la forme :
	\begin{semiverbatim}if (condition):
        < traitement des données >
    else:
        < traitement des données >\end{semiverbatim}
    le \verb|else| étant optionnel si on ne désire rien faire si la condition n'est pas vérifiée.
    
    La condition est une expression de type \verb|bool| (pour \textit{booléen}, nommé après le mathématicien et logicien George Boole). Une variable de type \verb|bool| prend deux valeurs seulement : \verb|True| et \verb|False|.
\end{frame}

\begin{frame}[fragile]
	Parmi les façons de construire des conditions booléennes, les plus courantes sont celles qui comparent les variables entre elles. Pour cela, on utilise les tests logiques de la table suivante:
	\begin{figure}
	\centering
	\begin{tabular}{|l|c|}\hline
	Égalité & \verb|x==y| \\ 
	Différent & \verb|x!=y| \\ 
	Inférieur ou égal & \verb|x<=y| \\ 
	Inférieur strictement & \verb|x<y| \\ 
	Supérieur ou égal & \verb|x>=y| \\
	Supérieur strictement & \verb|x>y| \\ \hline
	\end{tabular}
	\end{figure}
	\pause
	
	\begin{block}{Exemple}
	Définir une variable \verb|x| égale à $3$. Quel est le type de l'expression \verb|x==3| ? L'évaluer. Que donne l'évaluation de \verb|x==2| ? Celle de \verb|x<=4| ? Celle de \verb|x<3| ? Celle de \verb|x!=2| ?
	\end{block}
	\pause
	
	On peut combiner deux booléens \verb|a| et \verb|b| pour faire des opérations logiques en utilisant les opérateurs suivants:
	\begin{figure}
	\centering
	\begin{tabular}{|l|c|}\hline
	Négation & \verb|not(a)| \\
	Conjonction (<< et >>) & \verb|a and b| \\
	Disjonction (<< ou >>) & \verb|a or b| \\ \hline
	\end{tabular}
	\end{figure}
\end{frame}

\begin{frame}[fragile]
	\begin{block}{\textbf{Exercice 5}}
		Écrire un programme qui demande à l'utilisateur deux nombres $x$ et $y$, et affiche << \verb|Maximum de x et y = | >> suivi du maximum de $x$ et $y$ (le plus grand des deux).
	\end{block}
\end{frame}
\end{document}