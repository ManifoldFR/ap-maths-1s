\documentclass[10pt]{beamer}
\usepackage{geometry}
\usepackage{mathtools}
\usepackage{amssymb}
\usepackage{amsthm}
\usepackage{amsfonts}
\usepackage{graphicx}

\usetheme{Frankfurt}

\usepackage{fontspec}
\usepackage{polyglossia}
\setdefaultlanguage{french}

\renewcommand\epsilon\varepsilon
\renewcommand\phi\varphi
\newcommand{\N}{\mathbb N}
\newcommand{\Z}{\mathbb Z}
\newcommand{\R}{\mathbb R}
\newcommand{\Q}{\mathbb Q}
\newcommand{\CC}{\mathbb C}
\newcommand{\K}{\mathbb K}
\DeclarePairedDelimiter{\intcc}{[\![}{]\!]}

\title{Atelier d'informatique}
\subtitle{\textbf{Épisode II:} Structures de données}

\begin{document}

\begin{frame}[fragile]
	\titlepage
\end{frame}

\section{Introduction}

\begin{frame}[fragile]{Introduction}
	
	Dans ce chapitre, nous allons voir la notion de \textit{structure de données}, permettant de stocker et organiser un ensemble de données au sein d'un seul objet dans la mémoire de l'ordinateur, avec un type spécifique permettant de les traiter.
	\pause
	
	On verra principalement le type \verb|list| implémenté par défaut dans Python.
\end{frame}

\section{Listes Python}

\begin{frame}[fragile]{Listes Python}{Bases}
	Sous Python, une liste \verb|l| une collection d'éléments de la forme $e = (\verb|mem|, \verb|suivant|)$ où \verb|mem| est l'adresse mémoire d'une valeur, et \verb|suivant| un pointeur vers le successeur de $e$ au sein de la liste \verb|l|. Ainsi chaque élément contient où trouver la valeur correspondante et où trouver l'élément suivant.
	\pause
	
	Pour définir une liste en Python, plusieurs méthodes son possibles. On va commencer avec la définition d'une liste par explicitation : on énumère les éléments de la liste que l'on veut, entre des crochets \verb|[| et \verb|]|, et séparés par une virgule (\verb|,|).
	\pause
	
	Il n'est pas obligé d'entrer des valeurs du même type, même si cela n'est pas vraiment conseillé.
	\pause
	
	\begin{block}{\textbf{Exemple}}
		Dans la console, entrez l'instruction \verb|l = [0,1,2,3,4,5]|. Affichez \verb|l|.
		\pause
		
		Évaluez l'expression \verb|len(l)|. Que remarquez-vous ?
	\end{block}
	
	La fonction \verb|len| permet en effet d'évaluer la longueur d'une liste, comme elle servait à évaluer celle d'une chaîne de caractères.
\end{frame}

\begin{frame}[fragile]{Listes Python}{Accéder à un élément dans une liste}
	Pour accéder à un élément d'une liste, on utilise la syntaxe \verb|l[i]|, où \verb|i| est la position de l'élément de la liste qui nous intéresse... mais attention, on commence à compter à partir de zéro !
	\pause
	
	\begin{block}{\textbf{Exemple}}
		Que vaut \verb|l[0]| ? Tentez d'évaluer \verb|l[6]|.
	\end{block}
	\pause
	
	On peut même compter à rebours...
	\begin{block}{\textbf{Exemple}}
		Que vaut \verb|l[-1]| ? Et \verb|l[-2]| ?
	\end{block}
	\pause
	
	Pour obtenir une << sous-liste >>, ou un \textit{slicing}, des termes de \verb|i| à \verb|j-1|, on utilise la syntaxe \verb|l[i:j]|.\pause \,Pour ne prendre qu'un élément sur 2, on utilise \verb|l[i:j:2]| ou plus généralement \verb|l[i:j:k]| pour n'en prendre que un sur \verb|k|.\pause 
	
	Les valeurs par défaut (utilisées quand rien n'est précisé) de \verb|i|, \verb|j| et \verb|k| sont respectivement $0$, l'indice de fin de la liste, et $1$.
\end{frame}


\begin{frame}[fragile]{Listes Python}{Modifier une liste}
	Les listes Python sont des objets dits \textit{mutables} : on peut modifier les valeurs qu'ils contiennent.
	\pause
	
	En Python, on peut modifier un élément d'une liste en réaffectant la valeur comme on le ferait avec une variable.
	\begin{block}{\textbf{Exemple}}
		Entrez \verb|l[0] = "banane"| et affichez \verb|l|. Que remarquez-vous ?
	\end{block}
	\pause
	
	Mais, une liste n'est qu'une collection d'adresses mémoire et pointeurs ! 
	\begin{block}{\textbf{Exemple}}
	Copiez la liste \verb|l| dans une autre variable \verb|t|. Exécutez l'instruction \verb|t[0] = "Tartiflette"|. Affichez \verb|t| pour vérifier que l'affectation a été faite.
	\pause
	
	Maintenant, affichez \verb|l|. Que remarquez-vous ?
	\end{block}
	
	Il existe une parade, en faisant un \textit{slicing} : \verb|t = l[:]|. Mais ce n'est pas tout à fait idéal...
\end{frame}

\begin{frame}[fragile]{Listes Python}{Modifier une liste (suite)}
	Enfin, on peut ajouter une valeur à une liste en utilisant la méthode \verb|append|, qui s'utilise comme suit : \begin{semiverbatim}l.append(x)\end{semiverbatim}
	ajoute la valeur \verb|x| à la fin de la liste \verb|l|.
	
	\begin{block}{\textbf{Exemple}}
		Effectuez l'instruction \verb|l.append(18)|, et affichez \verb|l|. Vérifiez la longueur via \verb|len|.
	\end{block}
\end{frame}

\section{Boucles d'instructions}

\begin{frame}[fragile]{Boucles}{Introduction}
	Des fois, on veut exécuter plusieurs fois les mêmes instructions au sein d'un programme, et le nombre de fois peut soit être très grand, soit dépendre des variables en jeu.
	\pause
	
	En tout cas, on ne va pas réécrire plusieurs fois ces mêmes instructions. Pour cela, on utilise des boucles.
	\pause
	
	Python propose deux types de boucle. La boucle itérative, \verb|for|, s'exécutant en lisant les éléments d'une structure de donnée que l'on peut parcourir, telle une liste, et la boucle conditionnelle \verb|while| qui ne s'arrête que lorsqu'une condition booléenne à vérifier prend la valeur \verb|False|.
\end{frame}

\begin{frame}[fragile]{Boucles}{Boucle itérative}
	Sous Python, elle est implémentée comme suit:
	\begin{semiverbatim}for i in itérable:
    < traitement >\end{semiverbatim}
	où \verb|itérable| peut être une liste, par exemple, ou encore une chaîne de caractères, ou généralement tout objet ayant l'attribut \verb|'__iter'|, c'est-à-dire sur lequel on peut itérer.
	\pause
	
	\begin{block}{\textbf{Exemple}}
		\begin{itemize}[<+->]
		\item Écrire un programme qui affiche 20 fois << \verb|J'aime la tartiflette| >>. On peut générer un itérable d'entiers allant de 0 à \verb|N-1| via \verb|range(N)|, ou de façon équivalente \verb|range(0,N)|.
		
		\item Écrire un programme qui compte de 1 à 10.
		
		\item Écrire un programme qui ajoute les inverses des entiers de $1$ à \verb|N|, où \verb|N| est un entier demandé à l'utilisateur.
		\end{itemize}
	\end{block}
\end{frame}

\begin{frame}[fragile]{Boucles}{Boucle conditionnelle}
	Sous Python, elle est implémentée comme suit:\begin{semiverbatim}while (condition):
    < traitement >\end{semiverbatim}
	où \verb|condition| est un booléen. La boucle s'exécute tant que le booléen est vrai.
	Pour forcer la sortie de boucle, on peut utiliser l'instruction \verb|break| au sein du traitement.
	
	\begin{block}{\textbf{Exemple}}
		Écrire un programme qui demande un entier à l'utilisateur tant que celui-ci n'est pas égal à $42$.
	\end{block}
\end{frame}

\end{document}