\documentclass[10pt,xcolor={dvipsnames}]{beamer}
\usepackage{polyglossia}
\usepackage{mathtools, amsthm, amssymb, amsfonts}
\usepackage{graphicx}
\setdefaultlanguage{french}
\usepackage{listings}
\usepackage{lstautogobble}

\usefonttheme{professionalfonts}
\usepackage{unicode-math}

\setmathfont{Latin Modern Math}
%\setmathfont[range={it}]{Latin Modern Sans 10 Oblique}
\setmathfont[range=up]{Latin Modern Sans}
\setmathfont[range={\mathbb}]{xits-math.otf}
%\setmathfont{xits-math.otf}
\setmonofont[Mapping=tex-text,Scale=0.9]{Consolas}

\lstset{language=Python,
	escapechar=\%,
	basicstyle=\ttfamily,
	keywordstyle=\color{OliveGreen}\ttfamily,
	stringstyle=\color{red}\ttfamily,
	commentstyle=\color{brown}\ttfamily,
	morecomment=[l][\color{blue}]{\#},
	columns=fixed,
	tabsize=4,
	autogobble=true
	keepspaces=false
}

%\usetheme{Frankfurt}
\usetheme{Hannover}
\usecolortheme{dolphin}

\newcommand{\N}{\mathbb N}
\newcommand{\Z}{\mathbb Z}
\newcommand{\R}{\mathbb R}
\newcommand{\Q}{\mathbb Q}
\newcommand{\CC}{\mathbb C}
\newcommand{\K}{\mathbb K}
\DeclarePairedDelimiter{\intcc}{[\![}{]\!]}

\newcounter{Exercice}
\resetcounteronoverlays{Exercice}
\makeatletter
\newenvironment{exo}[1][\@nil]{
	\def\tmp{#1}
	\refstepcounter{Exercice}
	\begin{block}{\textbf{Exercice \theExercice}
			\ifx\tmp\@nnil
			%
			\else
			(#1)
			\fi
		}
	}{\end{block}}
\makeatother

\newcounter{Exemple}
\resetcounteronoverlays{Exemple}
\makeatletter
\newenvironment{exem}[1][\@nil]{
	\def\tmp{#1}
	\refstepcounter{Exemple}
	\begin{block}{\textbf{Exemple \theExemple}
			\ifx\tmp\@nnil
			%
			\else
			(#1)
			\fi
		}
	}{\end{block}}
\makeatother
\title{Atelier d'informatique}
\subtitle{\textbf{Épisode II:} Structures de données}

\begin{document}

\begin{frame}[fragile]
	\titlepage
\end{frame}

\section{Introduction}

\begin{frame}[fragile]{Introduction}
	
	Dans ce chapitre, nous allons voir la notion de \textit{structure de données}, permettant de stocker et organiser un ensemble de données au sein d'un seul objet dans la mémoire de l'ordinateur, avec un type spécifique permettant de les traiter.
	\pause
	
	On verra principalement le type \lstinline|list| implémenté par défaut dans Python.
\end{frame}

\frame{\tableofcontents}

\section{Structures conditionnelles}

\subsection{Sémantique}

\begin{frame}[fragile]
	\frametitle{Structures conditionnelles}
	\framesubtitle{Sémantique}
	On peut demander à un programme de traiter différemment ses données selon les valeurs des variables introduites. Pour cela, on utilise une structure conditionnelle, qui prend la forme :
	\begin{lstlisting}
	if (condition):
		< instructions >
	else:
		< instructions >
	\end{lstlisting}
	le \lstinline|else| étant optionnel si on ne désire rien faire si la condition n'est pas vérifiée.
	\pause
	
	La condition est une expression de type \lstinline|bool| pour \textit{booléen}, nommé après le mathématicien et logicien George Boole. Une variable de type \lstinline|bool| prend deux valeurs : \lstinline|True| (\textit{Vrai}) et \lstinline|False| (\textit{Faux}).
\end{frame}

\subsection{Tests de comparaison}

\begin{frame}[fragile]{Structures conditionnelles}{Tests de comparaison}
	Parmi les façons de construire des conditions booléennes, les plus courantes sont celles qui comparent les variables entre elles. Pour cela, on utilise les tests logiques de la table suivante:\pause
	\begin{figure}
		\centering
		\begin{tabular}{|l|c|}\hline
			Égalité & \lstinline|x==y| \\ 
			Différent & \lstinline|x!=y| \\ 
			Inférieur ou égal & \lstinline|x<=y| \\ 
			Inférieur strictement & \lstinline|x<y| \\ 
			Supérieur ou égal & \lstinline|x>=y| \\
			Supérieur strictement & \lstinline|x>y| \\ \hline
		\end{tabular}
	\end{figure}
	\pause
	
	\textbf{Remarque} Les comparaisons de type égalité fonctionnent sur toutes les variables (sauf celles de type \lstinline|None|, dans lequel cas on teste \lstinline|is None|). Les comparaisons de type inférieur/supérieur fonctionnent sur les nombres, ainsi que sur les chaînes de caractères. Dans le premier cas, c'est l'ordre usuel sur les nombres réels, et dans le deuxième cas l'ordre lexicographique (l'ordre des mots dans un dictionnaire).
\end{frame}


\begin{frame}
	\begin{exem}
		Définissez une variable \lstinline|x| égale à $3$. Quel est le type de l'expression \lstinline|x==3| ? Évaluez sa valeur. Que donne l'évaluation de \lstinline|x==2| ? Celle de \lstinline|x<=4| ? Celle de \lstinline|x<3| ? Celle de \lstinline|x!=2| ?
	\end{exem}

	\begin{exem}
		Définissez une variable \lstinline|x| contenant la chaîne de caractère \lstinline|"Wolfgang"|. Évaluez les booléens \lstinline|x > "Amadeus"| et \lstinline|x < "Mozart"|. Comparez \lstinline|x| et \lstinline|"Wagner"|.
	\end{exem}
\end{frame}


\subsection{Opérations sur les booléens}
\begin{frame}[fragile]{Structures conditionnelles}{Opérations sur les booléens}
	On peut combiner deux booléens \lstinline|a| et \lstinline|b| pour faire des opérations logiques en utilisant les opérateurs suivants:\pause
	\begin{figure}
		\centering
		\begin{tabular}{|l|c|}\hline
			Négation & \lstinline|not(a)| \\
			Conjonction (<< et >>) & \lstinline|a and b| \\
			Disjonction (<< ou >>) & \lstinline|a or b| \\ \hline
		\end{tabular}
	\end{figure}
	\pause
	
	\begin{exo}
		Écrire un programme qui demande deux nombres \lstinline|x| et \lstinline|y| à l'utilisateur et affiche s'ils sont tous les deux strictement positifs (c'est-à-dire s'ils sont $>0$).
	\end{exo}
\end{frame}

\begin{frame}[fragile]
	\begin{exo}
		Écrire un programme qui demande à l'utilisateur deux nombres $x$ et $y$, et affiche << \lstinline|Maximum de x et y = | >> suivi du maximum de $x$ et $y$ (le plus grand des deux).
	\end{exo}
\end{frame}

\section{Listes Python}

\subsection{Les bases}

\begin{frame}[fragile]{Listes Python}{Bases}
	Sous Python, une liste \lstinline|l| une collection d'éléments de la forme $e = ($\lstinline|mem|, \lstinline|suivant|$)$ où \lstinline|mem| est l'adresse mémoire d'une valeur, et \lstinline|suivant| un pointeur vers le successeur de $e$ au sein de la liste \lstinline|l|. Ainsi chaque élément contient où trouver la valeur correspondante et où trouver l'élément suivant.
	\pause
	
	Pour définir une liste en Python, plusieurs méthodes son possibles. On va commencer avec la définition d'une liste par explicitation : on énumère les éléments de la liste que l'on veut, entre des crochets \lstinline|[| et \lstinline|]|, et séparés par une virgule (\lstinline|,|).
	\pause
	
	Il n'est pas obligé d'entrer des valeurs du même type, même si cela n'est pas vraiment conseillé.
\end{frame}

\begin{frame}[fragile]
	\begin{exem}
		Dans la console, entrez l'instruction \lstinline|l = [0,1,2,3,4,5]|. Affichez \lstinline|l|.
		\pause
		
		Évaluez l'expression \lstinline|len(l)|. Que remarquez-vous ?
	\end{exem}
	
	La fonction \lstinline|len| permet en effet d'évaluer la longueur d'une liste, comme elle servait à évaluer celle d'une chaîne de caractères.
\end{frame}

\subsection{Accéder à un élément}

\begin{frame}[fragile]{Listes Python}{Accéder à un élément dans une liste}
	Pour accéder à un élément d'une liste, on utilise la syntaxe \lstinline|l[i]|, où \lstinline|i| est la position de l'élément de la liste qui nous intéresse... mais attention, on commence à compter à partir de zéro !
	\pause
	
	\begin{exem}
		Que vaut \lstinline|l[0]| ? Tentez d'évaluer \lstinline|l[6]|.
	\end{exem}
	\pause
	
	On peut même compter à rebours...
	\begin{exem}
		Que vaut \lstinline|l[-1]| ? Et \lstinline|l[-2]| ?
	\end{exem}
\end{frame}

\subsection{Modifier une liste}

\begin{frame}[fragile]{Listes Python}{Modifier une liste}
	Les listes Python sont des objets dits \textit{mutables} : on peut modifier les valeurs qu'ils contiennent.
	\pause
	
	En Python, on peut modifier un élément d'une liste en réaffectant la valeur comme on le ferait avec une variable.
	\begin{exem}
		Entrez \lstinline|l[0] = "Banane"| et affichez \lstinline|l|. Que remarquez-vous ?
	\end{exem}
	\pause
	
	Mais, une liste n'est qu'une collection d'adresses mémoire et pointeurs ! 
	\begin{exem}
	Copiez la liste \lstinline|l| dans une autre variable \lstinline|t|. Exécutez l'instruction \lstinline|t[0] = "Hegel"|. Affichez \lstinline|t| pour vérifier que l'affectation a été faite.
	\pause
	
	Maintenant, affichez \lstinline|l|. Que remarquez-vous ?
	\end{exem}
	
	Il existe une parade, en faisant un \textit{slicing} : \lstinline|t = l[:]|. Mais ce n'est pas tout à fait idéal...
\end{frame}

\begin{frame}[fragile]
	Enfin, on peut ajouter une valeur à une liste en utilisant la méthode \lstinline|append| : \lstinline|l.append(x)| ajoute la valeur \lstinline|x| à la fin de la liste \lstinline|l|.

	\begin{exem}
		Effectuez l'instruction \lstinline|l.append(18)|, et affichez \lstinline|l|. Vérifiez la longueur via \lstinline|len|.
	\end{exem}
	\pause
	
	Si vous souhaitez insérer un élément \lstinline|x| en position \lstinline|i|, utilisez \lstinline|l.insert(i,x)|.
	\pause
	
	Si vous voulez étendre la liste \lstinline|l1| avec les valeurs d'une deuxième liste \lstinline|l2|, utilisez \lstinline|l1.extend(l2)|.
	\pause
	
	La méthode \lstinline|l.sort()| trie la liste, la méthode \lstinline|l.reverse()| renverse son ordre.
\end{frame}

\subsection{Slicing}

\begin{frame}[fragile]{Listes dans Python}{\textit{Slicing}}
	Pour obtenir une << sous-liste >>, ou un \textit{slicing}, composé des termes de \lstinline|i| à \lstinline|j-1|, on utilise la syntaxe \lstinline|l[i:j]|.\pause \,Pour ne prendre qu'un élément sur 2, on utilise \lstinline|l[i:j:2]| ou plus généralement \lstinline|l[i:j:p]| pour n'en prendre que un sur \lstinline|p|.\pause 
	
	Les valeurs par défaut (utilisées quand rien n'est précisé) de \lstinline|i|, \lstinline|j| et \lstinline|k| sont respectivement $0$, l'indice de fin de la liste, et $1$.\pause
	
	On peut modifier en un coup la tranche \lstinline|[i:j:p]|, en l'affectant:
	\begin{lstlisting}l[i:j:p] = liste_de_valeurs
	\end{lstlisting}
	
	\begin{exo}
		\begin{itemize}[<+->]
			\item Définissez une liste \lstinline|li| de dix entiers.
			\item Extrayez-en la tranche \lstinline|[2:6]|. Affichez le résultat. Triez la tranche (via la méthode \lstinline|nom_liste.sort()|), et remettez le résultat dans \lstinline|li|. Affichez \lstinline|li|.
		\end{itemize}
	\end{exo}
\end{frame}

\subsection{Opérations sur les listes}

\begin{frame}[fragile]{Listes dans Python}{Opérations sur les listes}
	On peut ajouter deux listes ensemble: \lstinline|l1 + l2| produit la \textit{concaténation} des listes \lstinline|l1| et \lstinline|l2| (les éléments de la première suivis de ceux de la seconde).
	
	\begin{exem}
		Définissez deux listes \lstinline|l1| et \lstinline|l2|, affichez \lstinline|l1 + l2|.
	\end{exem}
\end{frame}

\section{Boucles d'instructions}
\subsection{Introduction}

\begin{frame}[fragile]{Boucles}{Introduction}
	Des fois, on veut exécuter plusieurs fois les mêmes instructions au sein d'un programme, et le nombre de fois peut soit être très grand, soit dépendre des variables en jeu.
	\pause
	
	En tout cas, on ne va pas réécrire plusieurs fois ces mêmes instructions. Pour cela, on utilise des boucles.
	\pause
	
	Python propose deux types de boucle. La boucle itérative, \lstinline|for|, s'exécutant en lisant les éléments d'une structure de donnée que l'on peut parcourir, telle une liste, et la boucle conditionnelle \lstinline|while| qui ne s'arrête que lorsqu'une condition booléenne à vérifier prend la valeur \lstinline|False|.
\end{frame}

\subsection{Boucle itérative}

\begin{frame}[fragile]{Boucles}{Boucle itérative}
Sous Python, elle est implémentée comme suit:
\begin{lstlisting}
for i in itérable:
    < traitement >
\end{lstlisting}
où \lstinline|itérable| peut être une liste, par exemple, ou encore une chaîne de caractères, ou généralement tout objet que l'on peut parcourir.
\pause

Le plus souvent, pour parcourir l'ensemble $\intcc{0,N-1}$ des entiers de $0$ à $N-1$, on utilise \lstinline|range(N)|. En général, \lstinline|range(i,j)| correspond à l'ensemble $\intcc{i,j-1}$ des entiers de $i$ à $j-1$.
\pause

	\begin{exo}
		\begin{itemize}[<+->]
		\item Écrire un programme qui affiche 20 fois \lstinline|"J'aime la tartiflette"|.
		
		\item Écrire un programme qui compte de 1 à 10.
		
		\item Écrire un programme qui ajoute les inverses des entiers de $1$ à \lstinline|N|, où \lstinline|N| est un entier demandé à l'utilisateur.
		\end{itemize}
	\end{exo}
\end{frame}

\subsection{Boucle conditionnelle}

\begin{frame}[fragile]{Boucles}{Boucle conditionnelle}
	Sous Python, elle est implémentée comme suit:
	\begin{lstlisting}
	while (condition):
	    < traitement >
	\end{lstlisting}
	où \lstinline|condition| est un booléen. La boucle s'exécute tant que le booléen est vrai.\pause
	Pour forcer la sortie de boucle, on peut utiliser l'instruction \lstinline|break| au sein du traitement.\pause
		
	\begin{exo}
		Écrire un programme qui demande un entier à l'utilisateur tant que celui-ci n'est pas égal à $42$.
	\end{exo}
\end{frame}

\end{document}