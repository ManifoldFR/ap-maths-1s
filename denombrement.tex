\documentclass[11pt]{article}
\usepackage[dvipsnames]{xcolor}
\usepackage[a4paper]{geometry}
\usepackage{mathtools,amssymb,amsthm}
\usepackage{polyglossia}
\usepackage{fontspec}
\usepackage{unicode-math}
\usepackage{titling}
\usepackage{textcomp}
\usepackage{tikz}
\usepackage[shortlabels]{enumitem}
\usepackage{mdframed}

\setdefaultlanguage{french}
\frenchspacing
\setmathfont{Latin Modern Math}
\setmathfont[range={\mathbb,\mathcal,\mathsf}]{xits-math.otf}

\newcommand{\N}{\mathbb N}
\newcommand{\Z}{\mathbb Z}
\newcommand{\R}{\mathbb R}
\newcommand{\Q}{\mathbb Q}
\newcommand{\CC}{\mathbb C}
\newcommand{\K}{\mathbb K}
\DeclareMathOperator{\Card}{\mathrm{Card}}
\DeclarePairedDelimiter{\zint}{[\![}{]\!]}

%% Theorem environments
\theoremstyle{definition}
%\newtheorem{defn}{Définition}[section]

\newmdtheoremenv[
	backgroundcolor=lightgray!50,
	linecolor=blue!80!black!60,
	linewidth=2pt,
	topline=false,
	rightline=false,
	bottomline=false
]{defn}{Définition}[section]

%\newtheorem{prop}[defn]{Proposition}

\newmdtheoremenv[
	backgroundcolor=BrickRed!15,
	linecolor=red!80!black!80,
	linewidth=2pt,
	topline=false,
	rightline=false,
	leftline=true,
	bottomline=false
]{prop}[defn]{Proposition}

\newtheorem{axio}[defn]{Axiome}
\newtheorem{exe}{Exemple}
\newtheorem{exo}{Exercice}

\theoremstyle{remark}
\newtheorem{rem}{Remarque}


\begin{document}

\begin{center}
	\hrulefill\\
    \vspace{6mm}
	\textsc{\LARGE Dénombrement}\\
    \vspace{3mm}
    \hrulefill
\end{center}
\vspace{1cm}

\begin{flushleft}
<< \textit{...combinatorics, a sort of glorified dice-throwing.} >> \hfill— Robert Kanigel
\end{flushleft}

\begin{flushleft}
	<< \textit{L'arithmétique, c'est être capable de compter jusqu'à vingt sans enlever ses chaussures.} >> \hfill — Walt Disney
\end{flushleft}

\begin{flushleft}
<< \textit{Tout ce dénombrement, madame, est inutile\\
Cent Hectors pourraient-ils me payer un Achille ?} >> \hfill — Jacques Pradon, \textit{La Troade}
\end{flushleft}


\tableofcontents

\section{Introduction}

\textit{Dénombrer}, c'est compter le nombre d'éléments qu'il y a dans un ensemble, le plus souvent défini par une propriété qu'il vérifie (par exemple, ses éléments pourraient être les parties de $\zint{1,12}$  qui ont $5$...)

Savoir dénombrer permet notamment de faire des calculs de probabilité plus compliquées à la main, et a des applications en physique ou encore en informatique.

Pour démarrer, deux exemples introductifs:

\paragraph{Problème 1} On considère une étagère sur laquelle se situent 4 livres différents.

\begin{figure}[h]
\centering
\begin{tikzpicture}
\draw (-1.4,0) rectangle (-0.4,2);
\draw (-0.4,0) rectangle (0.6,2.8);
\draw (0.6,0) rectangle (1.,2.6);
\draw (1.,0) rectangle (1.4,2.4);

\fill (-2,-0.2) rectangle (2,0);
\end{tikzpicture}
\end{figure}

De combien de façons peut-on ranger ces livres ?



\paragraph{Problème 2} On considère maintenant un sac de 10 billes différentes. De combien de façons peut-on constituer un paquet de 4 billes parmi les 10 ?

\section{Produit cartésien}

\begin{defn}[Couple de deux éléments]
	Soient $x$ et $y$ deux objets mathématiques. Le couple $(x,y)$ est la donnée de $x$ comme \textit{première composante} et de $y$ comme \textit{deuxième composante}.
	
	Il faut retenir la propriété caractéristique:
	Deux couples $(a,b)$ et $(x,y)$ sont égaux si et seulement si leurs composantes sont égales deux à deux:
	
	\[
	(a,b) = (x,y) \Longleftrightarrow (a=x\land b=y).
	\]
\end{defn}

\begin{defn}[Produit cartésien]
Soient $E$ et $F$ deux ensembles. On appelle \textit{produit cartésien de $E$ et de $F$}, et on note $E\times F$ (lu << $E$ croix $F$ >>) l'ensemble des couples $(x,y)$ où $x\in E$ et $y\in F$.
\end{defn}

\begin{rem}
En général, $E\times F\neq F\times E$ ! Pour avoir l'égalité, les deux ensembles doivent être égaux ou l'un des deux doit être vide.
\end{rem}

\begin{exe}\leavevmode
\begin{itemize}
\item $\R\times\R = \{(x,y)\ \mid\ x\in\R\,\text{et}\,y\in\R \}$
\item $(0,\pi)\in\R\times\R^*$ mais $(0,\pi)\not\in \R^*\times\R$.
\end{itemize}
\end{exe}

\begin{prop}
	Soient $E$ et $F$ des ensembles. Alors:
	\begin{itemize}
		\item $E\times\varnothing=\varnothing\times E=\varnothing$
		\item Si $E$ et $F$ sont \textbf{finis}, alors
		\[
		\Card(E\times F) = \Card(E)\Card(F). 
		\]
	\end{itemize}
\end{prop}

\begin{defn}[Généralisation]\leavevmode
	\begin{enumerate}
		\item Soient $a_1,\ldots,a_n$ des objets mathématiques. On définit le $n$-uplet $(a_1,\ldots,a_n)$ comme étant le couple ayant comme première composante le $n-1$-uplet $(a_1,\ldots,a_{n-1})$ et comme deuxième composante $a_n$. Les $n-1$-uplets ayant été définis de la même manière à partir des $n-2$-uplets.
		\item Soient $E_1,\ldots,E_n$ des ensembles, avec $n\in\N^*$. Le \textit{produit cartésien} des ensembles $(E_i)_{1\leq i\leq n}$ est l'ensemble des \textit{$n$-uplets}
		\[ (x_1,\ldots,x_n)
		\]
		où pour tout $i$ entre $1$ et $n$, $x_i\in E_i$.
	\end{enumerate}
\end{defn}

Lorsque tous les $E_i$ sont égaux à un même ensemble $E$, on notera
\[
\underbrace{E\times\ldots \times E}_{n\; \text{fois}} = E^n.
\]

\begin{exe}
	Ainsi, $\R\times\R = \R^2$.
\end{exe}

\begin{prop}
	Soient $E$, $F$ et $G$ trois ensembles. Par définition des $n$-uplets, on a 
	\[(E\times F)\times G = E\times(F\times G) = E\times F\times G. \]
\end{prop}


\section{Applications}

Dans cette section, on définira de manière générale la notion d'\textit{application} entre deux ensembles, une généralisation des fonctions que vous connaissez. $E$ et $F$ sont dans la suite deux ensembles.

\subsection{Généralités}

\begin{defn}
	Une application $f$ de $E$ vers $F$, notée $f:E\longrightarrow F$ est la donnée d'une partie $G$ de $E\times F$, appelée \textit{graphe de $f$}. Si $(x,y)\in G$, $y$ est appelé \textit{image de $x$ par $f$}, $x$ est appelé \textit{antécédent de $y$ par $f$}. De plus, $G$ doit vérifier la propriété suivante:
	pour tout $(x,y)\in G$, et pour tout $y'$ de $F$,
	\[
	(x,y')\in G \Rightarrow y=y'
	\]
	c'est-à-dire qu'un élément $x\in E$ admet au plus une image. On adopte la notation des fonctions: si $(x,y)\in G$, alors
	\[y=f(x).\]
	
	L'ensemble des applications de $E$ vers $F$ est noté $F^E$ ou encore \(\mathcal{F}(E,F)\).
\end{defn}


Bien sûr, la définition donnée ne garantit pas qu'une application $f:E\longrightarrow F$ est définie sur $E$ tout entier. On définit alors la notion d'\textit{ensemble de définition}:

\begin{defn}[Ensemble de définition]
	Soit $f$ une application de $E$ vers $F$. L'\textit{ensemble de définition} de $f$ est l'ensemble des $x\in E$ qui admettent une image par $f$, c'est-à-dire tels qu'il existe un $y\in F$ tel que $(x,y)\in G$.
\end{defn}

Des fois, il faut considérer non pas l'application $f$ mais une application définie sur une partie de $E$ prenant les mêmes valeurs que $f$.

\begin{prop}[Nombre d'applications]
	On suppose que $E$ et $F$ sont \textbf{finis}. Le nombre d'applications de $E$ vers $F$ est le cardinal de $\mathcal F(E,F)$, qui vaut
	\[
	\Card(\mathcal{F}(E,F)) = \Card(F)^{\Card(E)}.
	\]
\end{prop}

\subsection{Injection, surjection, bijection}

\begin{defn}
	Soit $f$ une application de $E$ vers $F$, que l'on suppose définie sur $E$ tout entier. Elle est dite:
	\begin{itemize}
		\item \textit{injective} si $f(x)=f(x')$ entraîne $x= x'$. Autrement dit, les éléments de $F$ n'admettent au plus qu'un antécédent.
		\item \textit{surjective} si tout $y\in F$ admet au moins un antécédent $x$ par $f$.
		\item \textit{bijective} si elle est à la fois injective \textbf{et} surjective. Autrement dit, tout élément $y$ de $F$ admet un \textit{et un seul} antécédent par $f$.
	\end{itemize}
\end{defn}

Si $f$ est bijective, alors pour tout $y\in F$ il existe un seul $x\in E$, que l'on peut noter $g(y)$ sans ambiguïté, tel que $y=f(x)$. L'application $g$ ainsi définie est appelée \textit{réciproque de $f$} et notée $f^{-1}$.


\subsection{Lien avec le cardinal}

\begin{defn}[Cardinal, version propre]
	L'ensemble $E$ est fini et est de cardinal $n\in\N^*$ s'il existe une bijection $\varphi:\zint{1,n}\longrightarrow E$, c'est-à-dire que l'on peut écrire
	$E=\{\varphi(1),\ldots,\varphi(n) \}.$
\end{defn}

\begin{prop}
	On suppose que $E$ et $F$ sont finis. Soit $f$ une application de $E$ vers $F$. Alors:
	\begin{itemize}
		\item Si $f$ est injective alors $\Card E\leq \Card F$.
		\item Si $f$ est surjective alors $\Card E \geq \Card F$.
		\item Si $f$ est bijective alors $\Card E = \Card F$: les ensembles ont le même nombre d'éléments.
	\end{itemize}
\end{prop}

Le dernier point est particulièrement utile dans le cadre du dénombrement: en effet, si on peut mettre en bijection l'ensemble $E$ dont on cherche le nombre d'éléments avec un ensemble $F$ dont on connaît bien le cardinal, alors on en déduit que $E$ a même cardinal de $F$.


\end{document}