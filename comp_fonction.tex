\documentclass{article}
\usepackage{polyglossia}
\usepackage[hmargin=4.4cm,vmargin=4cm]{geometry}
\usepackage{mathtools, amssymb, amsfonts}
\usepackage{amsthm}
\usepackage{titling}
\usepackage{listings}
\usepackage{graphicx}
\usepackage{fontspec}
\usepackage{multicol}
\usepackage[shortlabels]{enumitem}
\usepackage{comment}

\theoremstyle{definition}
\newtheorem{mydef}{Définition}[section]
\newtheorem{exo}{Exercice}
\newtheorem{exe}[mydef]{Exemple}

\theoremstyle{remark}
\newtheorem{rem}[mydef]{Remarque}

\theoremstyle{theorem}
\newtheorem{thm}[mydef]{Théorème}

\renewcommand\epsilon\varepsilon
\renewcommand\phi\varphi
\newcommand{\NN}{\mathbb N}
\newcommand{\ZZ}{\mathbb Z}
\newcommand{\RR}{\mathbb R}

\setdefaultlanguage{french}

\pretitle{\begin{center}\LARGE}
\title{\textsc{Composition de fonctions} }
\posttitle{\par\end{center}\vspace{-3.2em}}

\preauthor{\begin{center}\large}
\author{}
\postauthor{\par\end{center}}

\date{\today}

\begin{document}

\maketitle

\section{Généralités}

\begin{mydef}
Soient $f:D_f\longrightarrow\RR$ et $g:D_g\longrightarrow\RR$ deux fonctions réelles. On suppose qu'on ait que pour tout $x\in D_f$, $f(x)\in D_g$. 

Alors, pour tout $x\in D_f$, $g(f(x))$ existe. On peut donc définir la \textit{composée de $f$ par $g$}, notée $g\circ f$, par
\[
g\circ f:\begin{cases}
D_f &\longrightarrow \RR \\
x &\longmapsto g(f(x)) 
\end{cases}
\]
\end{mydef}

Ainsi, calculer $g\circ f(x)$ consiste en calculer $f(x)$ \textit{puis} appliquer $g$ au résultat.

\begin{exe}
Soient 
\[
f:\begin{cases}
\RR &\longrightarrow\RR \\
x &\longmapsto 1+x^2
\end{cases}
\quad\textrm{et} \quad
g:\begin{cases}
\RR^* &\longrightarrow \RR \\
x &\longmapsto \frac 1 x
\end{cases}.
\]
Alors la composée de $f$ par $g$ est
\[
g\circ f:\begin{cases}
\RR&\longrightarrow\RR \\
x &\longmapsto \dfrac{1}{1+x^2}
\end{cases}
\]
\end{exe}

\begin{rem}
Calculez la composée de $g$ par $f$, $f\circ g$: que remarquez-vous ?
\end{rem}

\begin{exo}
Dans chaque cas, dire si on peut définir, et le cas échéant calculer la composée de $f$ par $g$:
\begin{enumerate}
\item $f:\RR\longrightarrow\RR,\ x\longmapsto 1+x^2$ et $g$ la fonction racine carrée.
\item $f:\RR\longrightarrow\RR,\ x\longmapsto -x$ et $g$ la fonction racine carrée.
\item $f:{]0,+\infty[}\longrightarrow\RR,\ x\longmapsto \sqrt{x+1}$ et $g:\RR/\{-1;1\}\longrightarrow\RR,\ x\longmapsto \dfrac{x}{x^2-1}$
\item $f:D\longrightarrow\RR,\ x\longmapsto \dfrac{\sqrt{x-5}}{x^2-x-6}$, $g:{]0,+\infty[}\longrightarrow\RR,\ x\longmapsto \dfrac{1}{x^2}$, où il faut déterminer le domaine $D$ de $f$ (éventuellement le restreindre pour pouvoir composer...)
\end{enumerate}
\end{exo}

\section{Variations}

\begin{thm}[Sens de variation d'une composée]
Soient $f$ et $g$ deux fonctions telles que pour tout $x\in D_f$, $f(x)\in D_g$. Alors:\begin{itemize}
\item Si $f$ et $g$ ont même sens de variation alors $g\circ f$ est \textbf{croissante}.
\item Si $f$ et $g$ ont des sens de variation contraires alors $g\circ f$ est \textbf{décroissante}.
\end{itemize}
\end{thm}

\begin{exo}
En écrivant les fonctions suivantes comme des composées de fonctions simples dont vous connaissez les sens de variation, établir les variations de\begin{itemize}
\item $x\longmapsto \dfrac{1}{1+x^2}$ sur $\RR$.
\item $x\longmapsto |x|$ sur $\RR$.
\item $x\longmapsto \dfrac{1}{1-\sqrt{x-3}}$ sur son ensemble de définition, à préciser.
\end{itemize}
\end{exo}

\begin{exo}
Démontrer le théorème 2.1.
\end{exo}

\end{document}