\documentclass{article}
\usepackage{polyglossia}
\usepackage[hmargin=4.4cm]{geometry}
\usepackage{mathtools, amssymb, amsfonts}
\usepackage{amsthm}
\usepackage{multicol}
\usepackage{titling}
\usepackage[shortlabels]{enumitem}
\usepackage{comment}

\theoremstyle{definition}
\newtheorem{mydef}{Définition}[section]
\newtheorem{exe}{Exemple}[section]
\newtheorem{exo}{Exercice}


\renewcommand\epsilon\varepsilon
\renewcommand\phi\varphi
\newcommand{\N}{\mathbb N}
\newcommand{\Z}{\mathbb Z}
\newcommand{\R}{\mathbb R}

\setdefaultlanguage{french}

\pretitle{\begin{center}\LARGE}
\title{\textsc{Polynômes du second degré (1\textsuperscript{ère} S)} }
\posttitle{\par\end{center}\vspace{-3.2em}}

\preauthor{\begin{center}\large}
\author{}
\postauthor{\par\end{center}}

\date{\today}

\begin{document}

\maketitle

\begin{exo} Résoudre sur $\R$ (les 2,4,6 en mettant sous forme canonique):
    \begin{multicols}{2}
        \begin{enumerate}
            \item $3x^2-2x-1=0$
            \item $14x^2 + 3x - 2 = 0$
            \item $x^2 - 3x + 2 = 0$
            \item $-x^2 + 5x - 7 \geq 0$
            \item $\frac{2}{5}x^2 - \frac{3}{\sqrt{5}}x - 7 \leq -8$
            \item $x^3-x^2 \geq x^2 -x$
        \end{enumerate}
    \end{multicols}
\end{exo}

\begin{exo} Soit $f$ la fonction polynômiale
    \begin{equation*}
        \left\{\begin{array}{lcl}
            \R &\longrightarrow &\R \\
            x &\longmapsto & x^3 + x^2 - x + 2
        \end{array}\right.
    \end{equation*}
    On cherche à étudier le signe de $f$ sur $\R$. Pour cela, on va le factoriser.
    \begin{enumerate}
        \item Vérifier que $-2$ est racine de $f$.
        \item D'après un résultat que l'on admettra, il existe donc un trinôme $g$ tel que $f(x) = (x+2)g(x)$ pour tout réel $x$. Déterminer l'expression de $g$.
        \item Établir le tableau de signe de $f$.
    \end{enumerate}
\end{exo}

\begin{exo}
    Soit $f$ le polynôme de degré 2 défini par $f(x)\coloneqq -x^2+3x+2$. Résoudre sur $\R$ l'inéquation
    \[
    0\leq f(x) \leq 2.
    \]
\end{exo}

\end{document}
