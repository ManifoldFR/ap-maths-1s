\documentclass[11pt]{article}
\usepackage{geometry}
\usepackage{mathtools,amssymb,amsthm}
\usepackage{polyglossia}
\usepackage{fontspec}
\usepackage{unicode-math}
\usepackage{titling}
\usepackage[shortlabels]{enumitem}

\setdefaultlanguage{french}

%% Theorem environments
\theoremstyle{definition}
\newtheorem{defn}{Définition}[section]
\newtheorem{prop}[defn]{Proposition}
\newtheorem{axio}[defn]{Axiome}
\newtheorem{exe}{Exemple}


\theoremstyle{remark}
\newtheorem{rem}{Remarque}


\begin{document}

\begin{center}
	\textsc{\LARGE Ensembles, Logique}
\end{center}
\vspace{1cm}

\begin{flushright}
\textit{<< La logique est l'hygiène des mathématiques. >>} — André Weil\\
\textit{<< Tout ce qui est rare est super cher,\\
    Une Maserati bon marché est rare,\\
    Donc une Maserati bon marché, c'est super che >>} — Bruce Benamran
\end{flushright}

\section{Logique}

\subsection{Propositions, formules}

\begin{defn}[Proposition]
Une \textit{proposition} est la donnée d'une affirmation $P$. Elle peut prendre deux valeurs de vérité : le \textit{Vrai} ($V$) et le \textit{Faux} ($F$).
\end{defn}

\begin{exe}\leavevmode
\begin{enumerate}
\item << Aristote est mortel. >>
\item << Aristote est un chat. >>
\item << Les lois de la physique aristotélicienne sont vraies. >>
\end{enumerate}
\end{exe}

\begin{defn}[Formule propositionnelle]
On peut partir des propositions de base dont on dispose pour construire des \textit{formules}, en les combinant avec des symboles appelés \textit{connecteurs logiques}.

C'est donc la donnée d'une expression où sont articulées des propositions à l'aide de connecteurs. Elles sont en soi des propositions.
\end{defn}

On verra plus loin quels sont ces symboles qui permettent d'articuler des propositions pour créer une formule.


La logique usuelle obéit au principe suivant (on peut décider de ne pas l'adopter, ce qui donne lieu à d'autres systèmes de logique bizarres étudiés par des chercheurs en logique...):

\begin{axio}[Principe du tiers exclu]
Une proposition prend la valeur \textit{Vrai} ou la valeur \textit{Faux}, \textbf{jamais les deux}.
\end{axio}

\begin{defn}[Identité de deux propositions]
Deux formules $P$ et $Q$ sont dites \textit{équivalentes} ou \textit{identiques} si elles ont toujours même
\end{defn}

\subsection{Opérateurs logiques}

\begin{defn}[Négation]
Soir $P$ une proposition. On appelle \textit{négation de $P$} et on note $\neg P$ la proposition de valeur de vérité opposée à $P$: quand $P$ est vraie, $\neg P$ est fausse, et vice-versa. Sa table est donnée par
\begin{center}
  \begin{tabular}{|c|c|}\hline
  $P$ & $\neg P$ \\ \hline
  $V$ & $F$ \\ \hline
  $F$ & $V$ \\ \hline
  \end{tabular}
\end{center}
\end{defn}

\begin{prop}
Soit $P$ une proposition. Alors
\[\neg\neg P \equiv P.
\]
\end{prop}

\begin{exe}

\end{exe}






\end{document}



