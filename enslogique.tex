\documentclass[11pt]{article}
\usepackage[a4paper]{geometry}
\usepackage{mathtools,amssymb,amsthm}
\usepackage{polyglossia}
\usepackage{fontspec}
\usepackage{unicode-math}
\usepackage{titling}
\usepackage[shortlabels]{enumitem}

\setdefaultlanguage{french}
\frenchspacing

%% Theorem environments
\theoremstyle{definition}
\newtheorem{defn}{Définition}[section]
\newtheorem{prop}[defn]{Proposition}
\newtheorem{axio}[defn]{Axiome}
\newtheorem{exe}{Exemple}
\newtheorem{exo}{Exercice}

\theoremstyle{remark}
\newtheorem{rem}{Remarque}


\begin{document}

\begin{center}
	\textsc{\LARGE Ensembles, Logique}
\end{center}
\vspace{1cm}

\begin{flushright}
\textit{<< La logique est l'hygiène des mathématiques. >>} — André Weil\end{flushright}

\begin{flushright}
\begin{flushleft}
\textit{<< Tout ce qui est rare est super cher,\\
    Une Maserati bon marché, c'est rare,\\
    Donc une Maserati bon marché, c'est super cher >>}
\end{flushleft}
— Bruce Benamran
\end{flushright}

\begin{flushright}
<<La mathématique est une science dangereuse : elle dévoile les supercheries et les erreurs de calculs.>> — Galilée
\end{flushright}


\tableofcontents

\section{Logique}

\subsection{Propositions, formules}

\begin{defn}[Proposition]
Une \textit{proposition} est la donnée d'une affirmation $P$. Elle peut prendre deux valeurs de vérité : le \textit{Vrai} ($V$) et le \textit{Faux} ($F$).
\end{defn}

\begin{exe}\leavevmode
\begin{enumerate}
\item << Aristote est mortel. >>
\item << Aristote est un chat. >>
\item << Les lois de la physique d'Aristote sont vraies. >>
\end{enumerate}
\end{exe}

\begin{defn}[Formule propositionnelle]
On peut partir des propositions de base dont on dispose pour construire des \textit{formules}, en les combinant avec des symboles appelés \textit{connecteurs logiques}.

C'est donc la donnée d'une expression où sont articulées des propositions à l'aide de connecteurs. Elles sont en soi des propositions.
\end{defn}

On verra plus loin quels sont ces symboles qui permettent d'articuler des propositions pour créer une formule.



La logique usuelle obéit au principe suivant (on peut décider de ne pas l'adopter, ce qui donne lieu à d'autres systèmes de logique bizarres étudiés par des chercheurs en logique...):

\begin{axio}[Principe du tiers exclu]
Une proposition prend la valeur \textit{Vrai} ou la valeur \textit{Faux}, \textbf{jamais les deux}.
\end{axio}

\begin{defn}[Identité de deux propositions]
Deux formules $P$ et $Q$ sont dites \textit{équivalentes} ou \textit{identiques} si elles ont toujours même valeur de vérité, quelles que soient les valeurs de vérité des variables qui y interviennent. On note
\[P\equiv Q
\]
\end{defn}

\begin{exe}\leavevmode
<< Aristote n'est pas mortel >> $\equiv$ << Aristote est immortel >>
\end{exe}

\begin{rem}Vous remarquerez qu'on ne se préoccupe pas de la véracité de ces propositions... Juste qu'elles reviennent à dire la même chose.
\end{rem}

\begin{defn}[Table de vérité]
La table de vérité d'une formule est un tableau dans lequel sont listées les valeurs de vérité de cette formule en fonction de celles des propositions qui y interviennent.
\end{defn}






\subsection{Opérateurs logiques}

\subsubsection{Négation}

\begin{defn}[Négation]
Soir $P$ une proposition. On appelle \textit{négation de $P$} et on note $\neg P$ la proposition de valeur de vérité opposée à $P$: quand $P$ est vraie, $\neg P$ est fausse, et vice-versa. Sa table de vérité est donnée par
\begin{center}
  \begin{tabular}{|c|c|}\hline
  $P$ & $\neg P$ \\ \hline
  $V$ & $F$ \\ \hline
  $F$ & $V$ \\ \hline
  \end{tabular}
\end{center}
\end{defn}

\begin{prop}
Soit $P$ une proposition. Alors
\[\neg\neg P \equiv P.
\]
\end{prop}

\begin{exe}\leavevmode\begin{enumerate}
\item $\neg$<< Aristote est mortel >> $\equiv$ << Aristote est immortel >>
\item $\neg$<< Aristote est un chat >> $\equiv$ << Aristote n'est pas un chat >>
\item $\neg$<< J'aime les frites >> $\equiv$ << \hspace{7cm} >>
\end{enumerate}

\end{exe}

\newpage
\subsubsection{Conjonction}

\begin{defn}[Conjonction]
On appelle \textit{conjonction} et on note $\land$ l'opérateur dont la table est donnée par

\begin{table}[!ht]
\centering
\begin{tabular}{|c|c|c|}\hline
$P$ & $Q$ & $P\land Q$ \\ \hline
$V$ & $V$ & $V$ \\\hline
$V$ & $F$ & $F$ \\\hline
$F$ & $V$ & $F$ \\\hline
$F$ & $F$ & $F$ \\\hline
\end{tabular}
\end{table}

$P\land Q$ se lit << $P$ et $Q$ >>.
\end{defn}

\begin{exe}
<< Aristote est mortel >> $\land$ << Aristote est un chat >> est ...\\
<< Tous les hommes sont mortels >> $\land$ << Tous les hommes sont mortels >> $\equiv$ ...\\
<< Ce mammifère est un chat >>$\land $ << Ce mammifère n'est pas un chat >> est...
\end{exe}

\begin{rem} Observez que dans le dernier cas, on peut conclure sans rien vraiment savoir sur le mammifère en question !
\end{rem}

\begin{prop}
Soient $P,Q,R$ des propositions.
\begin{enumerate}
\item $(P\land Q)\land R \equiv P\land (Q\land R)$
\item $P\land Q \equiv Q\land P$
\item $P\land P\equiv P$
\item $P\land \neg P \equiv F$ (principe du tiers exclu)
\end{enumerate}

\end{prop}



\subsubsection{Disjonction}

\begin{defn}[Disjonction]
On appelle \textit{disjonction} et on note $\lor$ l'opérateur dont la table est donnée par
\begin{table}[ht]
\centering
\begin{tabular}{|c|c|c|}\hline
$P$ & $Q$ & $P\lor Q$ \\ \hline
$V$ & $V$ & $V$ \\\hline
$V$ & $F$ & $V$ \\\hline
$F$ & $V$ & $V$ \\\hline
$F$ & $F$ & $F$ \\\hline
\end{tabular}
\end{table}

$P\lor Q$ se lit << $P$ ou $Q$ >>.
\end{defn}

\begin{prop}
Soient $P,Q,R$ des propositions.
\begin{enumerate}
\item $(P\lor Q)\lor R \equiv P\lor (Q\lor R)$
\item $P\lor Q \equiv Q\lor P$
\item $P\lor P\equiv P$
\item $P\lor \neg P\equiv V$ (principe du tiers exclu, encore)
\end{enumerate}
\end{prop}



\subsubsection{Implication, équivalence}

\begin{defn}[Implication]
L'opérateur \textit{implication}, noté $\Rightarrow$, a pour table de vérité

\begin{table}[ht]
\centering
\begin{tabular}{|c|c|c|}\hline
$P$ & $Q$ & $P\Rightarrow Q$ \\ \hline
$V$ & $V$ & $V$ \\\hline
$V$ & $F$ & $F$ \\\hline
$F$ & $V$ & $V$ \\\hline
$F$ & $F$ & $V$ \\\hline
\end{tabular}
\end{table}

$P\Rightarrow Q$ se lit << $P$ implique $Q$ >>.
\end{defn}

\begin{defn}[Équivalence]
L'opérateur \textit{équivalence}, noté $\Leftrightarrow$, a pour table de vérité

\begin{table}[ht]
\centering
\begin{tabular}{|c|c|c|}\hline
$P$ & $Q$ & $P\Leftrightarrow Q$ \\ \hline
$V$ & $V$ & $V$ \\\hline
$V$ & $F$ & $F$ \\\hline
$F$ & $V$ & $F$ \\\hline
$F$ & $F$ & $V$ \\\hline
\end{tabular}
\end{table}

$P\Leftrightarrow Q$ se lit << $P$ équivaut à $Q$ >>
\end{defn}




\subsubsection{Quelques équivalences de formules}

\begin{prop}[Distributivités]\leavevmode
\begin{enumerate}
\item $P\land (Q\lor R) \equiv(P\land Q)\lor (P\land R)$
\item $P\lor(Q\land R)\equiv (P\lor Q)\land (P\lor R)$
\end{enumerate}

\end{prop}

\begin{prop}[Négation d'une formule, lois de De Morgan]\leavevmode
\begin{enumerate}
\item $\neg(P\land Q) \equiv (\neg P)\lor(\neg Q)$ (loi de De Morgan)
\item $\neg(P\lor Q) \equiv (\neg P)\land (\neg Q)$ (deuxième loi de De Morgan)
\item $\neg(P\Rightarrow Q)\equiv P\land\neg Q$.
\item $\neg(P\Longleftrightarrow Q) \equiv P\Longleftrightarrow (\neg Q) \equiv (\neg P) \Longleftrightarrow Q$
\end{enumerate}
\end{prop}

\begin{rem}
Le point 3. donne un moyen simple de contredire une implication.
\end{rem}

\begin{exe}
On peut montrer que l'affirmation << si le sol est mouillé alors il pleut >> est fausse. En effet, elle se réécrit << le sol est mouillé >> $\Rightarrow $ << il pleut >>, et comme on peut avoir << le sol est mouillé >> et << il ne pleut pas >> (il suffit d'arroser le sol quand il fait beau...), on a un contre-exemple.
\end{exe}

\begin{prop}\leavevmode
\begin{enumerate}
\item $P\Longrightarrow Q\equiv (P\Longrightarrow Q)\land(Q\Longrightarrow P)$ (principe de double implication)
\item $(P\Rightarrow Q)\equiv (\neg Q\Rightarrow \neg P)$ (principe de la contraposée)
\item $(P\lor Q)\Rightarrow R\equiv (P\Rightarrow R)\land (Q\Rightarrow R)$ (disjonction de cas)
\item $P\Rightarrow Q \equiv \neg P\lor Q$
\item $(P\land (P\Rightarrow Q))\Rightarrow Q\equiv V$ (\textit{modus ponens} 
\item $P\Rightarrow(Q\lor R) \equiv (P\land \neg Q)\Rightarrow R $
\end{enumerate}
\end{prop}

\begin{exo}[D]
Niez:\begin{enumerate}
\item ((A\Rightarrow B)\land C)\lor\neg B
\item (((A\Leftrightarrow B)\lor C)\Rightarrow B)\Leftrightarrow A
\end{enumerate}
\end{exo}

\subsection{Formules quantifiées}

Maintenant, on va apprendre à manier des variables dans des formules logiques.

\begin{defn}[Prédicat]

\end{defn}

\begin{defn}[Quantificateur universel]
Le quantificateur universel $\forall$, lu << pour tout >> est défini comme suit: le prédicat
\[ \forall x,P(x)\]
est vrai si la proposition $P(x)$ est vraie quelle que soit la valeur prise par $x$.
\end{defn}

\begin{defn}[Quantificateur existentiel]
Le quantificateur universel $\exists$, lu << il existe >> est défini comme suit: le prédicat
\[ \exists x,P(x)\]
est vrai si la proposition $P(x)$ est vraie pour un certain $x$. Dans ce cas, on peut se donner un $x$ qui vérifie $P(x)$, même si on ne peut pas l'expliciter...
\end{defn}

\end{document}



