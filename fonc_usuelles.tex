\documentclass{article}
\usepackage{polyglossia}
\usepackage[a4paper,hmargin=4cm]{geometry}
\usepackage{mathtools, amssymb, amsfonts}
\usepackage{fontspec}
\usepackage{amsthm}
\usepackage{titling}
\usepackage{listings}
\usepackage{graphicx}
\usepackage[hidelinks]{hyperref}
\usepackage{multicol}
\usepackage[shortlabels]{enumitem}
\usepackage{comment}

\setdefaultlanguage{french}

%% Theorem environments
\theoremstyle{definition}
\newtheorem{mydef}{Définition}[section]
\newtheorem{exo}{Exercice}
\newtheorem{exe}[mydef]{Exemple}

\theoremstyle{remark}
\newtheorem{rem}[mydef]{Remarque}

\theoremstyle{plain}
\newtheorem{thm}[mydef]{Théorème}
\newtheorem{prop}[mydef]{Proposition}
\newtheorem{cor}[mydef]{Corollaire}

%% Math commands
\renewcommand\epsilon\varepsilon
\renewcommand\phi\varphi
\newcommand{\N}{\mathbb N}
\newcommand{\Z}{\mathbb Z}
\newcommand{\R}{\mathbb R}
\DeclarePairedDelimiter{\abs}{\lvert}{\rvert}

%% Titling configuration
\pretitle{\begin{center}\LARGE}
\title{\textsc{Autour des fonctions usuelles}}
\posttitle{\par\end{center}\vspace{-3.2em}}

\preauthor{\begin{center}\large}
\author{}
\postauthor{\par\end{center}}

\date{\today}

\begin{document}

\maketitle

\section{Valeur absolue}

\begin{mydef}
Soit $x$ un nombre réel. On appelle valeur absolue de $x$ le réel \textit{positif}
\[
\abs{x} = \begin{cases}
x &\textrm{si }x\geq 0 \\
-x &\textrm{si }x<0
\end{cases}.
\]
Ainsi, $\abs x=0$ si et seulement si $x=0$... 
\end{mydef}

La valeur absolue de $x$ représente en fait \textit{la distance de $x$ à 0} sur la droite numérique. On peut donc représenter la distance entre deux nombres $x$ et $y$ sur la droite numérique par la quantité $\abs{x-y}$.

Ceci, et les propriétés sur cette fonction, la rend fondamentale pour décrire certaines propriétés des nombres réels, et étudier des fonctions réelles.

\begin{prop}\label{prop:absmax}
Pour tout $x\in\R$,
\begin{equation}
\abs{x} = \max(-x,x).
\end{equation}
En effet, la définition précédente fait que la valeur absolue de $x$ est toujours plus le plus grand nombre entre parmi $x$ et $-x$.
\end{prop}

\begin{exe}
Par exemple, on a $\abs{-2} = 2$ et $\max(-2,2) = 2$...
\end{exe}

\begin{prop}\label{prop:absroot}
Pour tout réel $x$, on a
\begin{equation}\abs x = \sqrt{x^2}.\end{equation}
\end{prop}

Cette deuxième représentation de la valeur absolue nous amène le résultat suivant :

\begin{prop}\label{prop:absmult}
Pour tous $x$ et $y$ réels,
\begin{equation}\abs{xy} = \abs x \abs y.\end{equation}
\end{prop}

On peut en donner deux démonstrations: une qui utilise \ref{prop:absroot} et une autre qui revient à la définition.

\begin{exe}
on a $\abs{-6} = 6$ d'une part, et d'autre part $-6 = -2\times 3$ puis $\abs{-2} \times \abs 3 = 2 \times 3 = 6$...
\end{exe}

La propriété suivante est probablement l'une des inégalités les plus importantes de toutes les mathématiques.

\begin{prop}[Inégalité triangulaire]
Pour tous $x,y\in\R$,
\begin{equation}
\abs{x+y} \leq \abs x + \abs y.
\end{equation}
\end{prop}

Elle se généralise de la manière suivante : elle reste toujours valable si on remplace les nombres $x$ et $y$ par des vecteurs $\vec u$ et $\vec v$, et la valeur absolue par la norme : $\|\vec u+\vec v\|\leq \|\vec u\| + \|\vec v\|$.

Encore, on peut en donner deux démonstrations, qui utilisent les deux représentations de la valeur absolue données dans les propositions \ref{prop:absmax} et \ref{prop:absroot}.

\section{Fonction partie entière}

\begin{mydef}
	Soit $x\in\R$. On appelle \textit{partie entière de $x$} et on note $\lfloor x\rfloor$ l'unique entier tel que
	\[ \lfloor x\rfloor \leq x < \lfloor x\rfloor + 1. \]
\end{mydef}

\begin{exe}\leavevmode
	\begin{itemize}
		\item $\lfloor 2\rfloor = 2$
		\item $\lfloor 3,14\rfloor = 3$
		\item $\lfloor -2,5\rfloor = -3$
	\end{itemize}
\end{exe}

\begin{exo}Tracer la courbe représentative de la fonction $x\longmapsto \lfloor x\rfloor$.\end{exo}

\begin{exo}
	Montrer que pour tous $x\in\R$ et $n\in\Z$,
	\[ \lfloor x+n\rfloor = \lfloor x\rfloor + n. \]
\end{exo}

\begin{exo}
	En déduire que pour tous $x,y\in\R$, on a l'inégalité suivante :
	\[ \lfloor x\rfloor + \lfloor y\rfloor \leq \lfloor x+y\rfloor. \]
\end{exo}

\begin{cor}
	Alors, pour tous $x\in\R$ et $n\in\N$,
	\[ n\lfloor x\rfloor \leq \lfloor nx\rfloor. \]
\end{cor}

\begin{mydef}
	Soit $x$ un réel. On appelle \textit{partie fractionnaire de $x$} et on note $\{x\}$ le réel de $[0,1[$ défini par:
	\[ \{x\} \coloneqq x - \lfloor x\rfloor. \]
\end{mydef}

\begin{exe}\leavevmode
	\begin{itemize}
		\item $\{2\} = 0$
		\item $\{3,14\} = 0,14$
		\item $\{-2,5\} = 0,5$
	\end{itemize}
\end{exe}

\begin{exo}
	Soit $f:\R\longrightarrow\R$ une fonction 1-périodique, c'est-à-dire que pour tout réel $x$, $f(x+1)=f(x)$.
	
	Montrer que pour tout réel $x$, $f(x)= f(\{x\})$.
\end{exo}

\section{Quelques inégalités}

\begin{exo}[Moyennes usuelles]
	Soient $x$ et $y$ deux réels avec $0<x\leq y$. On pose les réels suivants:
	\begin{align*}
		m &= \frac{x+y}{2}\text{ moyenne arithmétique} \\
		g &= \sqrt{xy}\text{ moyenne géométrique} \\
		h &= \frac{2}{\frac{1}{x}+\frac{1}{y}}\text{ moyenne harmonique}
	\end{align*}
	
	Montrer que $x\leq h\leq g \leq m\leq y$.
\end{exo}

L'inégalité suivante est encore plus fondamentale que l'inégalité triangulaire présentée précédemment (elle sert même à démontrer sa version générale). Si je m'en rappelle bien, on peut utiliser cette inégalité pour montrer une partie des lois de la physique (plus précisément, les équations d'\textsc{Euler-Lagrange}).

\begin{exo}[Inégalité de \textsc{Cauchy-Schwarz}]
	Soient $x_1,\ldots,x_n$ et $y_1,\ldots,y_n$ des nombres réels. Montrer que
	\[ x_1y_1 + \cdots + x_ny_n \leq \sqrt{x_1^2+\cdots x_n^2}\sqrt{y_1^2\ldots y_n^2}. \]
	
	\textit{Indication:} Poser $f$ le trinôme du second degré $f(t) = (x_1 +ty_1)^2 + \cdots + (x_n +ty_n)^2$ et calculer son discriminant.
\end{exo}

\begin{rem}
	L'inégalité se reformule ainsi: si $\vec u$ et $\vec v$ sont deux vecteurs, on a 
	\[ |\vec u \cdot \vec v| \leq \|\vec u\|\cdot\|\vec v\|. \]
\end{rem}

\begin{exo}
	En déduire que si $x_1,\ldots,x_n>0$ on a
	\[ (x_1+\cdots+x_n)\left(\frac{1}{x_1}+\cdots\frac{1}{x_n}\right)
	\geq n^2. \]
\end{exo}

\begin{exo}[Application: inégalité triangulaire généralisée]\leavevmode
	Soient $\vec u$ et $\vec v$ deux vecteurs. Montrer l'inégalité triangulaire généralisée
	\[
	\|\vec u + \vec v\| \leq \|\vec u \| + \|\vec v\|.
	\]
\end{exo}

\end{document}