\documentclass{article}
\usepackage{polyglossia}
\usepackage[a4paper]{geometry}
\usepackage{mathtools, amssymb, amsfonts}
\usepackage{fontspec}
\usepackage{amsthm}
\usepackage{titling}
\usepackage{listings}
\usepackage{graphicx}
\usepackage[hidelinks]{hyperref}
\usepackage{multicol}
\usepackage[shortlabels]{enumitem}
\usepackage{comment}

\setdefaultlanguage{french}

%% Theorem environments
\theoremstyle{definition}
\newtheorem{mydef}{Définition}[section]
\newtheorem{exo}{Exercice}
\newtheorem{exe}[mydef]{Exemple}

\theoremstyle{remark}
\newtheorem{rem}[mydef]{Remarque}

\theoremstyle{plain}
\newtheorem{thm}[mydef]{Théorème}
\newtheorem{prop}[mydef]{Proposition}

%% Math commands
\renewcommand\epsilon\varepsilon
\renewcommand\phi\varphi
\newcommand{\N}{\mathbb N}
\newcommand{\Z}{\mathbb Z}
\newcommand{\R}{\mathbb R}
\DeclarePairedDelimiter{\abs}{\lvert}{\rvert}

%% Titling configuration
\pretitle{\begin{center}\LARGE}
\title{\textsc{Autour des fonctions usuelles}}
\posttitle{\par\end{center}\vspace{-3.2em}}

\preauthor{\begin{center}\large}
\author{}
\postauthor{\par\end{center}}

\date{\today}

\begin{document}

\maketitle

\section{Valeur absolue}

\begin{mydef}
Soit $x$ un nombre réel. On appelle valeur absolue de $x$ le réel \textit{positif}
\[
\abs{x} = \begin{cases}
x &\textrm{si }x\geq 0 \\
-x &\textrm{si }x<0
\end{cases}.
\]
Ainsi, $\abs x=0$ si et seulement si $x=0$... 
\end{mydef}

La valeur absolue de $x$ représente en fait \textit{la distance de $x$ à 0} sur la droite numérique. On peut donc représenter la distance entre deux nombres $x$ et $y$ sur la droite numérique par la quantité $\abs{x-y}$.

Ceci, et les propriétés sur cette fonction, la rend fondamentale pour décrire certaines propriétés des nombres réels, et étudier des fonctions réelles.

\begin{prop}\label{prop:absmax}
Pour tout $x\in\R$,
\begin{equation}
\abs{x} = \max(-x,x).
\end{equation}
En effet, la définition précédente fait que la valeur absolue de $x$ est toujours plus le plus grand nombre entre parmi $x$ et $-x$.
\end{prop}

\begin{exe}
Par exemple, on a $\abs{-2} = 2$ et $\max(-2,2) = 2$...
\end{exe}

\begin{prop}\label{prop:absroot}
Pour tout réel $x$, on a
\begin{equation}\abs x = \sqrt{x^2}.\end{equation}
\end{prop}

Cette deuxième représentation de la valeur absolue nous amène le résultat suivant :

\begin{prop}\label{prop:absmult}
Pour tous $x$ et $y$ réels,
\begin{equation}\abs{xy} = \abs x \abs y.\end{equation}
\end{prop}

On peut en donner deux démonstrations: une qui utilise \ref{prop:absroot} et une autre qui revient à la définition.

\begin{exe}
on a $\abs{-6} = 6$ d'une part, et d'autre part $-6 = -2\times 3$ puis $\abs{-2} \times \abs 3 = 2 \times 3 = 6$...
\end{exe}

La propriété suivante est probablement l'une des inégalités les plus importantes de toutes les mathématiques.

\begin{prop}[Inégalité triangulaire]
Pour tous $x,y\in\R$,
\begin{equation}
\abs{x+y} \leq \abs x + \abs y.
\end{equation}
\end{prop}

Elle se généralise de la manière suivante : elle reste toujours valable si on remplace les nombres $x$ et $y$ par des vecteurs $\vec u$ et $\vec v$, et la valeur absolue par la norme : $\|\vec u+\vec v\|\leq \|\vec u\| + \|\vec v\|$.

Encore, on peut en donner deux démonstrations, qui utilisent les deux représentations de la valeur absolue données dans les propositions \ref{prop:absmax} et \ref{prop:absroot}.

\end{document}